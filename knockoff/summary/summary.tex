\documentclass[11pt]{article}
\input{\string~/.macros}
\usepackage[a4paper, total={7in, 9in}]{geometry}
\usepackage{bbm}
\usepackage{mathrsfs} % really cursive alphabets
\usepackage{graphicx}
    \graphicspath{{./assets}}
\usepackage{hyperref}
    \hypersetup{colorlinks=true, linktoc=all, linkcolor=blue, citecolor=red}
\usepackage[backend=bibtex,sorting=none]{biblatex}
\usepackage[margin=0cm]{caption}

% random variables
\newcommand\ry{\ensuremath{\mathsf{y}}}
\newcommand\rx{\ensuremath{\mathsf{x}}}
\newcommand\rb{\ensuremath{\mathsf{b}}} 
\newcommand\rc{\ensuremath{\mathsf{c}}}
\newcommand\rz{\ensuremath{\mathsf{z}}}
\newcommand\ru{\ensuremath{\mathsf{u}}}
\newcommand\rw{\ensuremath{\mathsf{w}}}
\newcommand\rpa{\ensuremath{\mathsf{pa}}}
\newcommand\rU{\ensuremath{\mathsf{U}}}
\newcommand\rbx{\ensuremath{\mathsf{\mathbf{x}}}}
\newcommand\rby{\ensuremath{\mathsf{\mathbf{y}}}}
\newcommand\rbu{\ensuremath{\mathsf{\mathbf{u}}}}

\newcommand\bbP{\ensuremath{\mathbbm{P}}}
\newcommand\bbQ{\ensuremath{\mathbbm{Q}}}


% boldsymbols
\renewcommand\bmu{\ensuremath{\boldsymbol{\mu}}}
\newcommand\bSigma{\ensuremath{\boldsymbol{\Sigma}}}
\newcommand\bgamma{\ensuremath{\boldsymbol{\gamma}}}
\newcommand\bomega{\ensuremath{\boldsymbol{\omega}}}
\newcommand\bnu{\ensuremath{\boldsymbol{\nu}}}


% optimization, classes of functions
\newcommand\scrF{\ensuremath{\mathscr{F}}}
\newcommand\scrS{\ensuremath{\mathscr{S}}}
\newcommand\scrP{\ensuremath{\mathscr{P}}}

% independence
\newcommand{\dperp}{\ensuremath{\perp\!\!\!\perp}}
\newcommand{\ndperp}{\ensuremath{\not\!\perp\!\!\!\perp}}

% operators
\newcommand{\prox}{\ensuremath{\mathsf{prox}}}

\addbibresource{variable_selection.bib}
\addbibresource{generative_GAN.bib}
\addbibresource{criterion_independence.bib}


\begin{document}

\section{Variable Selection}
 
Let $\ry$ be response variable and $\rx$ be explanatory variables or covariates. Given i.i.d. samples $(x,y)\in\R^p\times\R$ from the joint distribution $p_{\rx,\ry}$, we are interested in asking the question 
\begin{center}
    \textit{which of the many covariates $x_1,\cdots,x_p$ does the response $y$ depend on}?
\end{center}
assuming that the response does depend on a sparse set of variables. In general, we want to find a subset $\sS\subset[p]$ such that $\sS$ retains the relevant information in $\rx$ for making inference about $\ry$,
\begin{align}
    \text{maximize}_{\hat{\sS}\subset[p]} 
        \quad& \sQ(\hat{\sS}) \nonumber \\
    \text{subject to}
        \quad& |\hat{\sS}| \leq t \nonumber
\end{align}
where $\sQ(\cdot)$ quantifies the relevance of a feature subset to response. The choice of $\sQ(\cdot)$ should be 1) capable of detecting the desired functional dependence betweern the covariates and the response and 2) concentrated with respect to the underlying measure (generalize well to test data) \cite{songFeatureSelectionDependence2012}. Example of criteria $\sQ(\cdot)$ include leave-one-out error bound of SVM, mutual information $I(\rx;\ry)$, and Hilbert Space-based estimator like Hilbert-Schmidt Independence Criterion (HSIC) \cite{grettonMeasuringStatisticalDependence2005}.

\section{Variable Selection as Finding Markov Blancket}

In reality, we are interested in the causal relationship. However, quantifying causal effects requires interventions and not possible from purely observational data. A natural relaxation is to find covariates dependent (in a statistical sense) on the response, conditioned on all other observed features \cite{gimenezKnockoffsMassNew2019}. Formally, we want to find smallest $\sS\subset\pb{p}$ s.t.
\begin{align*}
    \ry \dperp \rx_{\setminus\sS} \mid \rx_{\sS} 
\end{align*} 
A natural interpretation is that the other variables $\rx_{\setminus\sS}$ do not provide additional information about $\ry$. If we think of $\sG$ as graph representing the joint distribution $p_{\rx,\ry}$, then $\sS$ is the markov blanket for node $\ry$. 
\begin{figure}[h!]
    \fig{assets/markov_blancket.png}{5cm}
    \caption{$\sS =\pc{\rx_1,\rx_3,\rx_4,\rx_7}$}
\end{figure}
We can pose the problem of finding the Markov blanket of $\ry$ as a multiple (independent) binary hypothesis test
\begin{align} 
    H_0^{(j)}:
        \ry \dperp \rx_j \mid \rx_{\setminus \pc{j}} 
    \quad\quad\text{for}\quad
        j = 1,\cdots,p
    \label{eq:test_markov_blanket}
\end{align}
Let $\sH_0 = \pc{j \mid H_0^{(j)} \text{ holds}}$ be the set of truly irrelevant covariates. In general, we are interested in maximizing true positives while controlling the number of false positives. Sometimes, a global threshold for p-values of each tests is overly conservative for large $p$, an alternative approach is to maximize \textit{power} while control \textit{false discovery rate} (\textsf{FDR}) \cite{benjaminiControllingFalseDiscovery1995}.
\begin{align}
    \text{maximize}_{\hat{\sS}\subset [p]}\quad
        &\E\pb{\frac{ |\hat{\sS} \setminus \sH_0| }{ |\hat{\sS}| }} \nonumber 
            \tag{power} \\
    \text{subject to}\quad
        &\textsf{FDR} := \E\pb{ \frac{ |\hat{\sS} \cap \sH_0| }{ \max\pc{|\hat{\sS}|,1} } } \leq q
    \label{eq:opt_fdr_control}
\end{align}
If $p_{\ry|\rx}(\cdot|x)$ assumes a parametric generalized linear model form, 
\[
    \E\pb{\ry|\rx} = g^{-1}(\eta)
    \quad\quad
    \eta = \beta_1 x_1 + \cdots + \beta_p x_p    
\]
Then by \cite{candesPanningGoldModelX2017}, testing for conditional independence (\ref{eq:test_markov_blanket}) is equivalent to the following test,
\[
    H_0^{(j)}:
        \beta_j = 0
    \quad\quad\text{for}\quad
        j = 1,\cdots,p
    \label{test_glm}
\]

\section{Model-X Knockoff}

Traditionally, $p_{\ry|\rx}$ is chosen to be in some parametric family, e.g. GLM, and variable selection with FDR control is performed by computing \& plugging p-values into the BHq procedure \cite{benjaminiControllingFalseDiscovery1995}. Recently, \cite{barberControllingFalseDiscovery2015,candesPanningGoldModelX2017} designed a \textit{knockoff} framework for performing variable selection on high-dimensional nonparametric models with finite sample guarantees over the constraints in (\ref{eq:opt_fdr_control}). The framework requires significant knowledge of $p_{\rx}$ and assumes nothing about the $p_{\ry|\rx}$. This might give way to performing reproducible and robust variable selection where the $p_{\ry|\rx}$ is parameterized by highly complex mappings, e.g. neural networks. In addition, modeling $p_{\rx}$ might be a suitable task for problems where we have large amount of unsupervised data, or we know a priori some structure about $p_{\rx}$, which are often the case for large scale machine learning applications. 
\begin{definition*} 
    $\tilde{\rx}$ is a model-X knockoff for $\rx$ if 
    \begin{align}
        \tilde{\rx} 
            &\dperp \ry \mid \rx 
            \label{eq:knockoff_cond_indep_of_response} \\
        (\rx,\tilde{\rx})_{\textsf{swap}(\sS)}
            &\overset{d}{=} (\rx,\tilde{\rx})
            \quad\quad\text{for any}\quad\quad
            \sS\subset[p]
            \label{eq:knockoff_swap_exchangeability}
    \end{align}
    where $(\cdot)_{\textsf{swap}(\sS)}$ swaps coordinates for all $j\in\sS $ with coordinate $j+p$ and leaves other coordinate unchanged.  (\ref{eq:knockoff_cond_indep_of_response}) is guaranteed if $\tilde{\rx}$ is constructed without knowledge of $\ry$. 
\end{definition*}
\begin{figure}[h!]
    \fig{assets/joint_covariates_knockoff.jpeg}{6cm}
    \caption{$\sG$ represents $p_{\rx,\tilde{\rx}}$. (\ref{eq:knockoff_cond_indep_of_response}) implies $\tilde{\rx}$ is not in the Markov blanket of node $\ry$ for a graph representing joint distribution $p_{\rx,\tilde{\rx}}$. (\ref{eq:knockoff_swap_exchangeability}) implies that the $\rx,\tilde{\rx}$ are pairwise exchangeable, i.e. taking any subset of (green) variables and swap them with their (red) knockoff creates an isomorphism of $\sG$, i.e. edges preserved from the perspective of swapped variables. In practice, we want $\rx_j,\tilde{\rx}_j$ be as independent as possible, i.e. no edge connecting $\rx_j,\tilde{\rx}_j$ for all $j\in[p]$}
\end{figure}
\noindent Intuitively, \textit{knockoffs} mimics the dependence structure as the original covariates $\rx$, while being invariant to \textsf{swap($\cdot$)} operation, and is independent of the response $\ry$. It serves as a control for evaluating how much of dependence on the response is due to dependence structure of other variables and how much of it is due to dependence with response $\ry$.
  

\subsection{Knockoff Procedure for LASSO}

Consider a linear Gaussian model $\ry = \beta^T\rx + \epsilon$ where $\rx\sim\sN(\bmu,\bSigma)$ and $\epsilon\sim\sN(0,1)$. Let $\bX \in\R^{n\times p}$ be any design matrix with $n>p$. The knockoff filtering procedure for computing variable selection with controlled FDR is given by
\begin{enumerate}
    \item (\textbf{Generate Knockoffs}) 
    To ensure exchangeability property (\ref{eq:knockoff_swap_exchangeability}), it must be
    \begin{align*}
        \begin{pmatrix}
            \rx \\
            \tilde{\rx}
        \end{pmatrix}            
            \sim \sN\left(
                \begin{pmatrix}
                    \bmu \\ 
                    \bmu \\
                \end{pmatrix}, 
                \begin{bmatrix}
                    \bSigma & \bSigma - \diag\pc{\bs} \\
                    \bSigma - \diag\pc{\bs} & \bSigma \\
                \end{bmatrix}\right)
    \end{align*}
    One way to construct knockoff is to sample $\tilde{x}$ from the conditional distribution \cite{candesPanningGoldModelX2017,gimenezKnockoffsMassNew2019},
    \begin{align*}
        \tilde{\rx} \mid (\rx = x)
            &\sim \sN\left(
                \bmu_{\tilde{\rx}|\rx}(x), \bSigma_{\tilde{\rx}|\rx}(x)
            \right) \\
        &\bmu_{\tilde{\rx}|\rx}(x) 
            = \left (I - \diag\pc{\bs}\bSigma^{-1}\right) x + \diag\pc{\bs}\bSigma^{-1} \bmu\\
        &\bSigma_{\tilde{\rx}|\rx}(x)
            = 2\diag\pc{\bs} - \diag\pc{\bs}\bSigma^{-1}\diag\pc{\bs}
    \end{align*}
    Alternatively, we can match empirical first and second moment \cite{barberControllingFalseDiscovery2015} and construct design for knockoff as 
    \begin{align*}
        \tilde{\bX}
            = \bX(\bI - \bSigma^{-1}\diag\pc{\bs}) + \tilde{\bU}\bC
    \end{align*}
    where $\tilde{\bU}\in\R^{n\times p}$ is orthonormal matrix whose column is orthogonal to $\bX$ and $\bC^T\bC = 2\diag\pc{\bs} - \diag\pc{\bs}\bSigma^{-1}\diag\pc{\bs}$ is a Cholesky decomposition.
    \item (\textbf{Compute Pairwise Statistics}) Compute Lasso for the coefficients
    \begin{align*}
        \hat{\bbeta}
            = \argmin_{\beta\in\R^{2p}}
                \frac{1}{2} \norm{\by - [\bX,\tilde{\bX}]\bbeta}_2^2 + \lambda \norm{\bbeta}_1
    \end{align*}
    Now we compute statistics $\rw$ for each pair of original and knockoff variables
    \begin{align*}
        w_j
            := w_j(\bX,\tilde{\bX},\by)
            = |\hat{\bbeta}_j| - |\hat{\bbeta}_{j+p}|
        \quad\quad\text{for}\quad\quad
            j = 1,\cdots,p
    \end{align*}
    which satisfy the coin-flip property. One important consequence is that for any $j\in\sH_0$, $\rw_j$ is a symmtric distribution about the origin \cite{barberControllingFalseDiscovery2015,candesPanningGoldModelX2017}.
    \item (\textbf{Compute Threshold for Statistics}) Given $q>0$ be target FDR, then let 
    \begin{align*}
        \tau_+
            &= \min\pc{t>0 \;\big|\; \widehat{\textsf{FDP}}(t) \leq q }
            \quad\text{where}\quad
            \widehat{\textsf{FDP}} = \frac{1 + \# \pc{ j \mid w_j \leq -t }}{\#\pc{j \mid w_j \geq t}} 
    \end{align*}
    \item (\textbf{Perform Test}) with threshold $\tau_+$,
    \begin{align*}
        \hat{\sS} = 
            \pc{j \mid w_j \geq \tau_+}
    \end{align*}
    ensures that $\textsf{FDR}\leq q$
\end{enumerate} 

\subsection{Current Problems}

\begin{enumerate}
    \item Generating knockoffs is a hard problem, there has been work to sample knockoffs using MCMC \cite{batesMetropolizedKnockoffSampling2019}, with generative model \cite{romanoDeepKnockoffs2018}, in particular with GAN \cite{jordonKnockoffGANGeneratingKnockoffs2018} and with latent variable models \cite{gimenezKnockoffsMassNew2019}.
    \item There has been a few papers that tries to analyze power of the knockoff framework assuming Gaussian data \cite{liuPowerAnalysisKnockoff2020}
    \item There has been some preliminary work on adopting multiple hypothesis testing and FDR control when $p_{\ry|\rx}$ is parameterized by neural networks, e.g. by MLP \cite{luDeepPINKReproducibleFeature2018}. The challenge here seems that it is not easy to maintain power of tests.
    \item It seems that currently, experimental datasets is either simulated or small in size of $n,p$. I have not found any attempts that tried this method on image datasets. Might be interesting to see how this fairs.
\end{enumerate}



 
\newpage
\printbibliography 



\end{document}