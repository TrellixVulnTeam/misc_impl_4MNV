\documentclass[11pt]{article}
\input{\string~/.macros}
\usepackage[a4paper, total={7in, 9in}]{geometry}
\usepackage{bbm}
\usepackage{mathrsfs} % really cursive alphabets
\usepackage{graphicx}
    \graphicspath{{./assets}}
\usepackage{hyperref}
    \hypersetup{colorlinks=true, linktoc=all, linkcolor=blue, citecolor=red}
\usepackage[backend=bibtex,sorting=none]{biblatex}
\usepackage[margin=0cm]{caption}

% random variables
\newcommand\ry{\ensuremath{\mathsf{y}}}
\newcommand\rx{\ensuremath{\mathsf{x}}}
\newcommand\rb{\ensuremath{\mathsf{b}}} 
\newcommand\rc{\ensuremath{\mathsf{c}}}
\newcommand\rz{\ensuremath{\mathsf{z}}}
\newcommand\ru{\ensuremath{\mathsf{u}}}
\newcommand\rw{\ensuremath{\mathsf{w}}}
\newcommand\rpa{\ensuremath{\mathsf{pa}}}
\newcommand\rU{\ensuremath{\mathsf{U}}}
\newcommand\rbx{\ensuremath{\mathsf{\mathbf{x}}}}
\newcommand\rby{\ensuremath{\mathsf{\mathbf{y}}}}
\newcommand\rbu{\ensuremath{\mathsf{\mathbf{u}}}}

\newcommand\bbP{\ensuremath{\mathbbm{P}}}
\newcommand\bbQ{\ensuremath{\mathbbm{Q}}}


% boldsymbols
\renewcommand\bmu{\ensuremath{\boldsymbol{\mu}}}
\newcommand\bSigma{\ensuremath{\boldsymbol{\Sigma}}}
\newcommand\bgamma{\ensuremath{\boldsymbol{\gamma}}}
\newcommand\bomega{\ensuremath{\boldsymbol{\omega}}}
\newcommand\bnu{\ensuremath{\boldsymbol{\nu}}}


% optimization, classes of functions
\newcommand\scrF{\ensuremath{\mathscr{F}}}
\newcommand\scrS{\ensuremath{\mathscr{S}}}
\newcommand\scrP{\ensuremath{\mathscr{P}}}

% independence
\newcommand{\dperp}{\ensuremath{\perp\!\!\!\perp}}
\newcommand{\ndperp}{\ensuremath{\not\!\perp\!\!\!\perp}}

% operators
\newcommand{\prox}{\ensuremath{\mathsf{prox}}}

\addbibresource{variable_selection.bib}


\begin{document}

\section{Variable Selection}
 
Let $\ry$ be response variable and $\rx$ be explanatory variables or covariates. Given i.i.d. samples $(x,y)\in\R^p\times\R$ from the joint distribution $p_{\rx,\ry}$, we are interested in asking the question 
\begin{center}
    \textit{which of the many covariates $x_1,\cdots,x_p$ does the response $y$ depend on}?
\end{center}
assuming that the response does depend on a sparse set of variables. In reality, we are interested in the causal relationship. However, quantifying causal effects requires interventions and not possible from purely observational data. A natural relaxation is to find covariates dependent (in a statistical sense) on the response, conditioned on all other observed features \cite{gimenezKnockoffsMassNew2019}. Formally, we want to find smallest $\sS\subset\pb{p}$ s.t.
\begin{align*}
    \ry \dperp \rx_{\sS} \mid \rx_{\setminus\sS}
\end{align*} 
A natural interpretation is that the other variables $\rx_{\setminus\sS}$ do not provide additional information about $\ry$. If we think of $\sG$ as graph representing the joint distribution $p_{\rx,\ry}$, then $\sS$ is the markov blanket for node $\ry$. 
\begin{figure}[h!]
    \fig{assets/markov_blancket.png}{5cm}
    \caption{$\sS =\pc{\rx_1,\rx_3,\rx_4,\rx_7}$}
\end{figure}
We can pose the problem of finding the Markov blanket of $\ry$ as a multiple binary hypothesis test
\begin{align} 
    H_0^{(j)}:
        \ry \dperp \rx_j \mid \rx_{\setminus \pc{j}} 
    \quad\quad\text{for}\quad
        j = 1,\cdots,p
    \label{eq:test_markov_blanket}
\end{align}
Let $\sH_0 = \pc{\rx_j \mid H_0^{(j)} \text{ holds}}$ be the set of truly irrelevant covariates. In general, we are interested in maximizing true positives while controlling the number of false positives. Sometimes, a global threshold for p-values of each tests is overly conservative for large $p$, an alternative approach is to maximize \textit{power} while control \textit{false discovery rate} (FDR) \cite{benjaminiControllingFalseDiscovery1995}.
\begin{align}
    \text{maximize}_{\hat{\sS}\subset [p]}\quad
        &\E\pb{\frac{ |\hat{\sS} \setminus \sH_0| }{ |\hat{\sS}| }} \nonumber \\
    \text{subject to}\quad
        &\E\pb{ \frac{ |\hat{\sS} \cap \sH_0| }{ \max\pc{|\hat{\sS}|,1} } } \leq q
    \label{eq:opt_fdr_control}
\end{align}
If $p_{\ry|\rx}(\cdot|x)$ assumes a parametric generalized linear model form, 
\[
    \E\pb{\ry|\rx} = g^{-1}(\eta)
    \quad\quad
    \eta = \beta_1 x_1 + \cdots + \beta_p x_p    
\]
Then by \cite{candesPanningGoldModelX2017}, testing for conditional independence (\ref{eq:test_markov_blanket}) is equivalent to the following test,
\[
    H_0^{(j)}:
        \beta_j = 0
    \quad\quad\text{for}\quad
        j = 1,\cdots,p
    \label{test_glm}
\]

\section{Model-X Knockoff}

Traditionally, $p_{\ry|\rx}$ is chosen to be in some parametric family, e.g. GLM, and variable selection with FDR control is performed by computing \& plugging p-values into the BHq procedure \cite{benjaminiControllingFalseDiscovery1995}. Recently, \cite{barberControllingFalseDiscovery2015,candesPanningGoldModelX2017} designed a \textit{knockoff} framework for performing variable selection on high-dimensional nonparametric models with finite sample guarantees over the constraints in (\ref{eq:opt_fdr_control}). The framework requires significant knowledge of $p_{\rx}$ and assumes nothing about the $p_{\ry|\rx}$. This might give way to performing reproducible and robust variable selection where the $p_{\ry|\rx}$ is parameterized by highly complex mappings, e.g. neural networks. In addition, modeling $p_{\rx}$ might be a suitable task for problems where we have large amount of unsupervised data, or we know a priori some structure about $p_{\rx}$, which are often the case for large scale machine learning applications.





\newpage
\printbibliography 




\end{document}